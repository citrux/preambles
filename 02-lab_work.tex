\documentclass[a4paper]{article}
\usepackage[utf8]{inputenc}
\usepackage[english,russian]{babel}

\usepackage{amssymb,amsfonts,amsmath}
\usepackage{enumerate}
\usepackage{indentfirst}
\usepackage{array}

\usepackage{graphicx}
\graphicspath{{plots/}, {images/}}%путь к рисункам

\usepackage{pscyr}
\usepackage{multirow}

\makeatletter
\renewcommand{\@biblabel}[1]{#1.}  % Заменяем библиографию
\makeatother                       % с квадратных скобок на точку.

\usepackage{geometry} % Меняем поля страницы
\geometry{left=2cm}% левое поле
\geometry{right=1.5cm}% правое поле
\geometry{top=2cm}% верхнее поле
\geometry{bottom=2cm}% нижнее поле

\renewcommand{\theenumi}{\arabic{enumi}}
\renewcommand{\labelenumi}{\arabic{enumi}.}
\renewcommand{\theenumii}{.\arabic{enumii}}
\renewcommand{\labelenumii}{\arabic{enumi}.\arabic{enumii}.}
\renewcommand{\theenumiii}{.\arabic{enumiii}}
\renewcommand{\labelenumiii}{\arabic{enumi}.\arabic{enumii}.\arabic{enumiii}.}

\renewcommand{\phi}{\varphi}
\renewcommand{\theta}{\vartheta}
\renewcommand{\epsilon}{\varepsilon}
\renewcommand{\arraystretch}{1.2}
\newcommand{\midnum}[1]{\langle#1\rangle}

\newcolumntype{C}[1]{>{\centering\arraybackslash}m{#1\textwidth}}

\usepackage{titlesec}

\titleformat{\section}[block]
{\normalfont\large\bfseries\center}{}{.5em}{}

\titleformat{\subsection}[block]
{\normalfont\bfseries\center}{}{.5em}{}

\usepackage{setspace}
\onehalfspacing
\usepackage{caption}
\captionsetup{labelsep=period}
\usepackage{tocvsec2}
\pagestyle{empty}
