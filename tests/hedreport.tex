\documentclass[14pt,final,titlepage]{hedreport}
\usepackage[russian]{babel}
\usepackage[utf8]{inputenc}
\usepackage[complex,derivative,shortcuts,vectors,environments]{hedmaths}
\usepackage[column]{hedfeatures}
\usepackage[nuclear]{hedphysics}

\begin{document}
	\begin{proposition}
		Вороне где-то Бог послал кусочек сыру...
	\end{proposition}
	\begin{comment}
		наверное TCP/IP пакетом
	\end{comment}
	\begin{lemma}
		если \( g_1(x), g_2(x) \in R[x] \), то \( C_{g_1g_2} = C_{g_1} C_{g_2} \)
	\end{lemma}
	\begin{solution}
		\( \nucleus{1}{2}{L} - \nucleus{2}{1}{M} = ? \)
	\end{solution}
	\begin{proof}
		http://goo.gl/iySIs
	\end{proof}
	\[
		\dder{}{t}\left( \pder{x}{\kappa} \right) = 
			\ppder{\divergence{\vec{r}}}{\phi}\cdot\rotor{\Gamma} + 
			\frac{\Re{x}}{\Im{\psi^3}} - \pcder{x}{\alpha}{\beta}
	\]
	\begin{table}[ht]
		\centering
		\caption{<<Пример таблицы с использованием центрирования и фиксации размера>>}
		\begin{tabular}{|C{.2}|C{.2}|C{.17}|C{.18}|C{.15}|}
			\hline
			Объект & Вид & Контрольный уровень & Прибор & Результат \\ \hline
			Рабочее место у манипулятора в операторной & Измерение мощности дозы & 
				1 мкЗв/час & ДРГ-01Т1 МКС-01Р-01 & 0,08\( \pm \)0,04 мкЗв/час \\ \hline
			Захваты манипулятора, поверхности подвижного стола & Контроль р/а 
				загрязнённости методом мазков & Отсутствие снимаемых загрязнений &
				МКС-01Р-01 & отсутствует \\ \hline
		\end{tabular}
	\end{table}
\end{document}